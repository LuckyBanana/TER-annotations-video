\documentclass[12pt,a4paper]{article}
\usepackage[latin1]{inputenc}
\usepackage[francais]{babel}
\usepackage[T1]{fontenc}
\usepackage{amsmath}
\usepackage{amsfonts}
\usepackage{amssymb}
\usepackage{url}
\usepackage{hyperref}
\title{TER - Annotations Vid�o}
\begin{document}
\maketitle
\section*{Sujet}
R�sum� : Ce projet mont� en collaboration avec l'UFRSTAPS (�tudiants L3/M1 en judo) a pour fonction de co-concevoir et r�aliser une application web d'annotations de vid�os de combats de judo.
La dur�e d'un combat de judo est de 5min, il est constitu� d'une quinzaine de s�quences de 20sec de travail, entrecoup�es de phases de 10sec de pauses command�es par l'arbitre. Une annotation, r�alis�e par des combattants et/ou des �tudiants, consiste � identifier des �v�nements et des phases marquantes qui vont servir au d�briefing ou � la formation des �l�ves et des combattants.

Cette application est compos�e de trois modules. Avec le premier module, l'utilisateur devra rep�rer les s�quences du combat en pointant un logigramme des actions possibles ; dans le second module,il s'agit de rep�rer et annoter des �v�nements particuliers de ces s�quences aux yeux des combattants et/ou des annotateurs. Le module trois permettra de d�battre collectivement en partie en pr�sence et en partie sur le web des annotations afin de faire progresser le jugement de tous les intervenants au d�bat (arbitres, entraineurs, pr�parateurs physiques, coaching, athl�tes et �tudiants).
\section*{Documentations}
\subsection*{Web S�mantique}
Web 3.0
\newline
\url{http://jplu.developpez.com/tutoriels/web-semantique/introduction/}

\section*{Technologies}
\subsection*{Android}
Android SDK
\newline
\url{http://developer.android.com/develop/index.html}

\subsection*{Play Framework}
Framework pur le d�veloppement d'applications WEB
\newline
\url{http://www.playframework.org/}
\newline
\url{http://www.playframework.org/documentation/2.0.4/Home
}
\newline
\url{http://linsolas.developpez.com/articles/java/play/guide/
}
\subsection*{JavaFx}
Plateforme pour le d�veloppement de RIA en JAVA
\newline
\url{http://en.wikipedia.org/wiki/Rich_Internet_Application
}
\newline
\url{http://en.wikipedia.org/wiki/JavaFX
}
\newline
\url{http://docs.oracle.com/javafx/
}
\newline
\url{http://www.programmez.com/tutoriels.php?tutoriel=54
}
\subsection*{VLC Plugin}
Lecture de vid�os (client web)
\newline
\url{http://wiki.videolan.org/Hacker_Guide/How_To_Write_a_Module}

\subsection*{Serveur}
Architecture WEB
\newline
\url{http://fr.wikipedia.org/wiki/Representational_State_Transfer
}
\newline
\url{http://www.xfront.com/REST-Web-Services.html
}
\newline
\url{http://www.ics.uci.edu/~fielding/pubs/dissertation/rest_arch_style.htm
}
\end{document}